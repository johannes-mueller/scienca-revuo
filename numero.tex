
\documentclass[full]{srevuo}

%\setcounter{page}{49}

\renewcommand*{\TitlePageArt}{%
}

\begin{document}

\SciencaRevuoTitolo{%
  jaro=201X,
  kajero=X,
  numero=XXX,
  volumo=XX,
}

\part*{Pri ISAE}
\bigskip
{
\small
La Internacia Scienca Asocio Esperanta (ISAE) fondiĝis en la jaro 1906
kaj estas la unua esperantlingva asocio, en kiu fakuloj el ĉiuj sciencaj
kampoj, profesiaj aŭ amatoraj, kaj interesatoj pri sciencaj aferoj unuiĝis al
mondvasta scienca asocio, uzanta la internacian lingvon Esperanto en scienco
kaj teĥniko.

La laŭstatuta celo de la Asocio estas la divastiĝo de la Internacia Lingvo en
ĉiujn sciencajn kaj teĥnikajn sferojn, kaj samtempe progresigi la pretecon ĉe
la publiko por akceptado de Esperanto.  Ligilo inter la internacilingvaj
fakuloj estas nia asocia organo Scienca Revuo (SR), daŭrigo de revuo fondita en
1904 kiel Internacia Scienca Revuo. Ĝi aperas kvarfoje en la jaro kun
originalaj artikoloj en Esperanto de aŭtoroj de ĉiuj naciecoj kaj kun sciigoj
el plej diversaj kampoj de sciado.
\par\bigskip\noindent
\fbox{\parbox{\textwidth}{\small
    \textbf{Membrokotizoj}:\par\medskip\noindent
    \parbox{0.3\linewidth}{
      \begin{tabular}{>{\hspace{-\tabcolsep}}lr}
        \hiderowcolors
        A-landanoj & 8~€/jaro \\
        B-landanoj & 4~€/jaro \\
        C-landanoj & 1~€/jaro \\
      \end{tabular}}
    \hfill
    \parbox{0.65\linewidth}{%
      Tiu kotizo inkluzivas la ricevon de la plej freŝaj enhavoj de \srevuo en
      cifereca formo. La abono de la jara volumo en libra formo kostas
      kromajn 12 €.
    } \par\medskip
    {\footnotesize
      \textit{A-landoj:} ĉiuj landoj krom B- kaj C-landoj;
      \textit{B-landoj:} iamaj socialismaj landoj de Eŭropo, ĉiuj landoj de Azio 
      krom Japanio kaj Korea respubliko;
      \textit{C-landoj:} Afriko kaj Latina Ameriko}
    \par\bigskip\noindent
    UEA-konto: \textbf{isae-z} 
  } 
}
\par\bigskip\noindent
\srevuo estas enpaĝigata per la tipografia sistemo \LaTeX, kiu
estas vaste uzata de sciencistoj. La tiparo uzata
nomiĝas \textit{Palatino}. Desegnaĵoj kaj diagramoj estas ĉefe
faritaj per la programaĵo TikZ. Propra dokumenta klaso estis
programita kiu definas la tipografian dizajnon kaj aŭtomatigas kelkajn
partojn de la enpaĝigo.
}
\vfill
\noindent
\fcolorbox{black}{black!20!white}{%
  \parbox{\textwidth}{\sffamily\small
    \textbf{La estraro de ISAE:} \par\medskip\noindent%
    \begin{tabular}[l]{>{\hspace{-\tabcolsep}\raggedright\itshape}llll}
      \hiderowcolors
      José Antonio Vergara & CL & Prezidanto & \EMail{joseantonio.vergara@gmail.com} \\
      Johannes Mueller     & DE & Scienca Revuo & \EMail{Johannes.Mueller@esperanto.de} \\
      Mélanie Maradan      & CH & Terminologio & \EMail{maradan@uea.org} \\
      Francesco Maurelli   & UK & Eksteraj rilatoj & \EMail{fran.mau@gmail.com}\\
      %Claude Labetaa       & FR & Rilatoj kun ILEI &
      %\EMail{labetaa.claude@wanadoo.fr}
      Alga Guernieri & IT & Membroadministrado & \EMail{alga_g@hotmail.it} \\
    \end{tabular}
    \par\bigskip\noindent
    %\parbox{0.9\textwidth} 
      \par\medskip\noindent
  %   \parbox{0.25\textwidth}{%
  %     Poŝta adreso: \\
  %     \textit{%
  %       Scienca Revuo \\ c/o Johannes Mueller \\ Solitudestraße 49/4 \\ 71638
  %       Ludwigsburg \\ Germanio}}
        Poŝta adreso: \\
        \textit{%
          Scienca Revuo $\cdot$ 
          c/o Johannes Mueller $\cdot$ Solitudestraße 49/4 $\cdot$ 
          71638 Ludwigsburg $\cdot$ Germanio}
  }
}


\part{Pri tiu ĉi eldono}
%\SRinput{antauxparolo}

\part{ISAE informas}

\part{Scienco kaj Esperanto}

\part{El la scienca mondo}

\part{Scienco kaj socio}



\part{Scienca eĥo}
%\SRinput{scienca-ehho}

\tableofcontents
\end{document}

%%% Local Variables: 
%%% mode: latex
%%% LaTeX-command: "xelatex --shell-escape"
%%% TeX-engine: xetex
%%% End: 
